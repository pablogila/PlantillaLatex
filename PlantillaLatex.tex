% PlantillaLatex v2023.02.18 by Pablo Gila Herranz
% Encuentra la versión más reciente en:
% https://github.com/pablogila/PlantillaLatex/
% Que aproveche  :D


\documentclass[12pt,a4paper]{article}


%%%%%%%%%%%%%%%%%%%%%%%%%%%%%%%%%%%%%%
%%%%%%%%%%  CUSTOM COMMANDS  %%%%%%%%%
%%%%%%%%%%%%%%%%%%%%%%%%%%%%%%%%%%%%%%
% Puedes crear tus propios comandos, para ahorrarte tiempo al escribir
% Por ejemplo, para escribir el nombre de un elemento químico puedes hacer:
\newcommand{\MA}{MAPbI$_3$}
% O para poner imágenes más rápidamente:
\newcommand{\NEWPIC}[3]{
\begin{figure}[H]
    \centering
    \includegraphics[width=#2]{fotos/#1}
    \caption{#3}
    \label{#1}
    \end{figure}
    }

%%%%%%%%%%%%%%%%%%%%%%%%%%%%%%%%%%
%%%%%%%%%%%  PAQUETES  %%%%%%%%%%%
%%%%%%%%%%%%%%%%%%%%%%%%%%%%%%%%%%
% Aquí pones los paquetes que vayas a usar en el documento.
% Es recomendable comentar los paquetes que necesitas, para que no se carguen innecesariamente.

%%%%%%%%%%  TEXTO Y MATES  %%%%%%%%

%\renewcommand{\familydefault}{\sfdefault} % Para cambiar la fuente por defecto. Por ejemplo, por rmdefault (roman, por defecto en LaTeX), sfdefault (serif), ttdefault(monospace), etc.
\usepackage[spanish,es-tabla]{babel} % Ponemos es-tabla para que en la tabla con label diga Tabla en vez de Cuadro
\usepackage[T1]{fontenc} % Necesario para escribir > <
\usepackage{amsmath,amssymb,amsfonts,latexsym,cancel} % Paquetes matemáticos
\usepackage{mathtools} % Dividir ecuaciones con begin{multlined} parte1\\parte2 etc o bien con {split}, aunque queda más feo. Las llaves y corchetes dan problemas, hay que ponerlas por separado con \Bigg\{
\usepackage{bm} % Comando \bm{}, negritas en textos matemáticos
\usepackage{slashed} % si usas el comando \slashed puedes poner símbolos matemáticos tachados en ecuaciones
\usepackage{xcolor} % Texto en colores con \textcolor{red}{TEXTO}. Más info y otros comandos: https://www.overleaf.com/learn/latex/Using_colours_in_LaTeX

%%%%%%%%%% BIBLIOGRAFÍA %%%%%%%%%%%

%\usepackage[sorting=none]{biblatex}
%\addbibresource{refsTFM.bib} % tu .bib de referencia, si es que tienes tus referencias con biblatex

%%%%%%%%%%  IMÁGENES  %%%%%%%%%%%%%

\usepackage{graphicx} % Para poner imágenes
\usepackage{float} % Para poner la imagen o tabla donde estamos con una [H]
%\usepackage{subfigure} % Poner varias imágenes juntas, comando \subfigure
%\usepackage{epstopdf} % Para usar fotos en formato .eps
\usepackage{wrapfig} % Para poner imágenes a un lateral del texto

%%%%%%%%%%  COLUMNAS  %%%%%%%%%%%%

%\usepackage{array} % Para usar m{ancho}, centrando verticalmente el texto de tablas
%\usepackage{longtable} % Hacer tablas largas
%\newcolumntype{E}{>{$} c <{$}} % Hemos creado un nuevo tipo de columna. ahora, al poner E (letra de nuestra elección) la columna correspondiente estará en modo matemático. DEPENDE DEL PACKAGE ARRAY.
%\setcounter{MaxMatrixCols}{40} % Matrices con más columnas

%%%%%%%%%%%  ENLACES  %%%%%%%%%%%%

\PassOptionsToPackage{hyphens}{url} % Para que los enlaces no se salgan del borde. Poner \sloppy\url{}
\usepackage[colorlinks=true,linkcolor=black,urlcolor=blue,citecolor=blue]{hyperref} % Hipervínculos, así como enlazar partes, citas, etc del mismo documento. Para enlaces: \href{https://www.youtube.com/}{Youtube} o bien \url{}


%%%%%%%%%%%%%%%%%%%%%%%%%%%%%%%%%%%%%%%%%%%%%
%%%%%%%%%  FORMATO DE LAS HOJAS  %%%%%%%%%%%%
%%%%%%%%%  edita esto con tus datos  %%%%%%%%
%%%%%%%%%%%%%%%%%%%%%%%%%%%%%%%%%%%%%%%%%%%%%

\usepackage{multicol} % Para dividir el texto en columnas
\usepackage{titlesec} % Para cambiar el formato de las secciones
\titleformat{\section}[block]{\large\bfseries\centering}{\thesection.}{1mm}{} % Centra las secciones y las deja bonitas. REQUIERE TITLESEC
\titleformat{\subsection}[block]{\bfseries\centering}{\thesubsection.}{1mm}{} % Centra las subsecciones y las deja bonitas. REQUIERE TITLESEC
\usepackage[left=2cm,right=2cm,top=2.5cm,bottom=3cm]{geometry} % Limitar márgenes. Puede ponerse también con lmargin, rmargin.
\parindent =0cm % Cambiar todas las sangrías
\usepackage{fancyhdr} % Un formato guay para papers
\pagestyle{fancy} % Que use dicho formato
\fancyhead{} % Eliminar definiciones previas 
\fancyhead[l]{TITULO} % AQUÍ PONES EL TÍTULO DE TU TRABAJO
\fancyhead[r]{Universidad de Nosedonde} % AQUÍ PONES TU UNIVERSIDAD
\fancyfoot{} % Eliminar definiciones previas
\fancyfoot[r]{\thepage} % El comando \thepage pone la página
\fancyfoot[l]{John Marston} % AQUÍ PONES TU NOMBRE
\renewcommand{\headrulewidth}{0.9pt} % Cambiar grosor línea arriba, o eliminarla (0pt)
\renewcommand{\footrulewidth}{0.5pt} % Crear y editar línea de abajo


%%%%%%%%%%%%%%%%%%%%%%%%%%%%%%%%%%%%%%%%%%%%%
%%%%%%%%%%  PÁGINA DE TÍTULO  %%%%%%%%%%%%%%%
%%%%%%%%%%  edita esto con tus datos  %%%%%%%
%%%%%%%%%%%%%%%%%%%%%%%%%%%%%%%%%%%%%%%%%%%%%

\begin{document}
\begin{titlepage} % Primera página aparte con el título
\begin{center} % Para centrar el texto
\vspace*{2\baselineskip} % Avanza filas. Si está al principio, hay que poner *
\hrule height 3pt
\vspace*{0.5\baselineskip}
{\Huge \textbf{Técnicas experimentales en Brujería III}}\\[0.1cm] % AQUÍ PONES LA ASIGNATURA
{\large \textbf{UNIVERSIDAD DE NOSEDONDE}} % AQUÍ PONES TU UNIVERSIDAD
\vspace*{0.5\baselineskip}
\hrule
\vspace*{6\baselineskip}
\includegraphics[width=7cm]{fotos/escudo}\\ % AQUÍ PONES TU FOTO DE PORTADA
\vspace*{4.5\baselineskip}
{\Huge \textbf{TÍTULO\\Y TÍTULO:\\[0.5cm]MÁS TÍTULO\\}} % AQUÍ PONES EL TÍTULO DE TU TRABAJO
\vfill % Desplaza hasta el final de la página
\textit{John Marston}\\ % AQUÍ PONES TU NOMBRE
%\today % Muestra la fecha
\end{center}
\end{titlepage}


%%%%%%%%%%%%%%%%%%%%%%%%%%%%%%%%
%%%%%%%%%%%  ÍNDICE  %%%%%%%%%%%
%%%%%%%%%%%%%%%%%%%%%%%%%%%%%%%%

\tableofcontents
\thispagestyle{empty} % Quitar el formato en esta página
\newpage
\setcounter{page}{1} % Que no empiece la primera página en el índice


%%%%%%%%%%%%%%%%%%%%%%%%%%%%%%%%%%%%%%%
%%%%%%%%%%%%%%  ABSTRACT  %%%%%%%%%%%%%
%%%%%%%%%%  Qué vas a hacer?  %%%%%%%%%
%%%%%%%%%%%%%%%%%%%%%%%%%%%%%%%%%%%%%%%

\renewcommand{\abstractname}{Abstract} % Para que lo llame Abstract, no Resumen
\addcontentsline{toc}{section}{Abstract} % Para mostrar el abstract en el índice. Esta línea ha de ponerse junto al contenido a listar.

\begin{abstract}
En el presente estudio se tratará de hacer algo y este es el abstract
\end{abstract}


%%%%%%%%%%%%%%%%%%%%%%%%%%%%%%%%%%%%%%%%%%%%%%%%
%%%%%%%%%%%%%%%%  A ESCRIBIRRR  %%%%%%%%%%%%%%%%
%%%%%%%%%%%%%%%%%%%%%%%%%%%%%%%%%%%%%%%%%%%%%%%%

%%%%%%%%% (pero antes... un truquito!) %%%%%%%%%
\iffalse
Este comando sirve para comentar
automáticamente párrafos enteros de texto,
como por ejemplo este mismo.
Puede ser útil para ocultar un borrador
del documento, things to-do, etc.
Para finalizar el comentario,
acabar con el comando:
\fi
%%%%%%%%%%%%%%%%%%%%%%%%%%%%%%%%%%%%%%%%%%%%%%%%

\begin{multicols}{2} % Ponemos 2 columnas. Hay que cerrarlo al final. Si queremos que no empiece una columna hasta acabar la anterior, ponemos \begin{multicols*}


\section{Introducción}

\subsection{Fundamento teórico}

Planck\footnote{footnote de Planck} hizo el dibujo \eqref{etiqueta}.

\begin{figure}[H]
\centering
\includegraphics[width=3cm]{fotos/escudo}
\caption{explicación dibujo \cite{referencia}}
\label{etiqueta}
\end{figure}

Para poner fórmulas en medio del texto se hace ${\vartriangle}\varepsilon$ y para ponerlas separadas se hace como en la ecuación \eqref{ecuacion}

\begin{equation}
\Delta\varepsilon=h\nu
\label{ecuacion}
\end{equation} % {equation*} si no queremos numerarla

Y por acá una ecuación en dos líneas con el comando multlined o split por si son demasiado largas:

\begin{equation*}
\begin{multlined}
C=\frac{\tau}{R}=\frac{1.738\cdot10^{-4}s}{550\Omega}\\
=3.16\cdot10^{-7}F=316nF
\end{multlined}
\end{equation*}

Ojito a esto: MATRICES

\begin{equation*}
\begin{pmatrix}
J_Q\\
I
\end{pmatrix}
=
\begin{pmatrix}
L_{11} & L_{12}\\
L_{21} & L_{22}
\end{pmatrix}
\begin{pmatrix}
\frac{\Delta T}{T^2}\\
\frac{\Delta \epsilon}{T}
\end{pmatrix}
\end{equation*}

Ahora una cosa chunga que probablemente no uses nunca pero como de repente un día lo necesites igual te cuesta encontrarlo: con el comando slashed, del paquete slashed, podemos tachar símbolos matemáticos en ecuaciones, por ejemplo:

\begin{equation*}
\mathcal{L}=[\overline{\psi}_r^f(x)(i\slashed\partial-m_f)\psi_r^f(x)+\text{cosaschungas}]
\end{equation*}

\section{Realización experimental}

\subsection{Montaje del experimento}
dibujitos noseque

\subsection{Consideraciones previas}
\label{consideraciones}

\begin{itemize}
\item elemento de lista 1
\item elemento de lista 2
\end{itemize}

\subsection{Realización del experimento}
blablabla
\section{Medidas y análisis de datos}
\subsection{Interpretación de resultados}
El experimento ha sido un éxito
\subsection{Fuentes de errores}
Pues hay muchos errores

\end{multicols} % Se acaban las dobles columnas


\section{Conclusiones}

Las conclusiones son blablabla. POR CIERTO!! Aprovecho que aquí en las conclusiones ya hemos acabado con el multicols, para contarte que podemos poner ecuaciones divididas en dos columnas usando precisamente multicols, en caso de que nuestro paper ocupe todo el ancho de la hoja y queramos aprovechar el espacio poniendo dos ecuaciones al lado:

\begin{multicols}{2}
\begin{equation}
    \alpha_{m}=\frac{0.1(V_{m}+35)}{1-\exp(\frac{-(V_{m}+35)}{10})}
    \label{ec:alpha_m}
\end{equation}

\begin{equation}
    \beta_{h}=\frac{1}{1+\exp(-0.1(V_{m}+30))}
    \label{ec:beta_h}
\end{equation}
\end{multicols}

Ah, y otra cosa que te puede interesar. Se puede poner una imagen a un lado del texto, con el paquete \textbf{wrapfig}:

\begin{wrapfigure}{l}{0.3\textwidth}
\centering
\includegraphics[width=2cm]{fotos/escudo.png}
\caption{Caption}
\end{wrapfigure}

Esto es un ejemplo de una imagen al lateral de un texto utilizando el paquete wrapfig.\\
En el código de latex, primero insertamos la figura con el comando begin\{wrapfigure\}, y seguidamente introducimos el texto que queremos que se sitúe al lateral de la imagen.\\
Podemos editar la orden del wrapfigure modificando el tamaño que ocupará la imagen, por ejemplo, en este caso está puesto 0.4, pero podría ser 0.5, etc. Luego el texto sigue por donde debería ir en cuanto se acaba la figura, mira creo que ya debería haber acabado la figura. ¿ves? funciona perfectamente ole ole lo caracole un saludo chavales.\\[0.5cm]
POR CIERTO, os acordáis de que al principio habíamos definido comandos personalizados para poner ecuaciones o imágenes?? pues ahora los usamos, mira:\\
Por un lado, la fórmula química: \MA \\
Por otro, vamos a meter una foto rápidamente con nuestro custom command:
\NEWPIC{escudo}{5cm}{descripción de la imagen}


%%%%%%%%%%%%%%%%%%%%%%%%%%%%%%%%%%%%
%%%%%%%%%%  BIBLIOGRAFÍA  %%%%%%%%%%
%%%%%%%%%%%%%%%%%%%%%%%%%%%%%%%%%%%%

% Para documentos con muchas referencias, como un TFG o un paper, es recomendable usar un gestor de referencias, como Zotero o Mendeley.
% En serio, te puede salvar semanas de desesperación.
% Desde Zotero puedes gestionar tus referencias con bibtex o biblatex (recomendado), te saca un archivo .bib que pones junto al .tex y te pone las referencias solas... pero el cómo hacer ese archivo .bib es otra movida, te lo googleas por tu cuenta.


\newpage
%\renewcommand{\refname}{Bibliografía} % Para que ponga eso en vez de referencias
\addcontentsline{toc}{section}{Referencias} % Para mostrar las referencias en el índice. Esta línea ha de ponerse junto al contenido a listar.


%%%%%%%%%%%%%%%%%%%%%%%%%%%%%%%%%%%%%%%
% Si usamos biblatex (RECOMENDADO) podemos printeaar la bibliografía con el siguiente comando, sin más vueltas de tuerca.
% IMPORTANTE haber añadido arriba del todo los paquetes de biblatex!!
% En caso de usar biblatex, se puede borrar el \addcontentsline{toc}{section}{Referencias}.
% Meter la bibliografía es ultra fácil con biblatex, solo tienes que poner:

%\printbibliography


%%%%%%%%%%%%%%%%%%%%%%%%%%%%%%%%%%%%%%%
% Si quisiera usar bibtex a secas, que ya viene integrado en LaTeX, sin usar paquetes extra:

%\bibliographystyle{IEEEtran} % Estilo
%\bibliography{refsTFM} % documento .bib


%%%%%%%%%%%%%%%%%%%%%%%%%%%%%%%%%%%%%%%
% Y si quisiera meter la bibliografía a mano, que es lo que seguramente haga la mayoría de la gente leyendo esto, pues es un coñazo monumental pero que sepas que se hace así:

\begin{thebibliography}{99} % 99 para que no de problemas de limitación

\bibitem{referencia} Albert Einstein, Un punto de vista heurístico sobre la producción y transformación de la luz, \textit{Annalen der Physik} \textbf{17}, 132-148 (1905)

\bibitem{plantilla} Puedes encontrar la última versión de esta plantilla en \url{https://github.com/pablogila/PlantillaLatex}

\bibitem{tutorial} Si quieres aprender \LaTeX te recomiendo echarle un ojo a los videotutos de \textit{Héctor Misael Bacilio Navarro} con los que aprendió el menda: \url{https://youtube.com/playlist?list=PLKRmVEXGjGWcUu0vv1wGNWI_xAtE_yiE_}

\end{thebibliography}


\end{document}