\documentclass[12pt,a4paper]{article}
\usepackage[spanish,es-tabla]{babel} %Ponemos es-tabla para que en la tabla con label diga Tabla en vez de Cuadro
\usepackage[T1]{fontenc} %Parece necesario para escribir > <
\usepackage{amsmath,amssymb,amsfonts,latexsym,cancel} %Paquetes matemáticos

\usepackage{graphicx} %Para poner imágenes
\usepackage{float} %Para poner la imagen o tabla donde estamos con una H
%\usepackage{subfigure} %Poner varias imágenes juntas, comando \subfigure
%\usepackage{epstopdf} %Para usar fotos en formato .eps

%\usepackage{array} %Para usar m{ancho}, centrando verticalmente el texto de tablas
%\usepackage{longtable} %Hacer tablas largas
%\newcolumntype{E}{>{$} c <{$}} %Hemos creado un nuevo tipo de columna. ahora, al poner E (letra de nuestra elección) la columna correspondiente estará en modo matemático. DEPENDE DEL PACKAGE ARRAY.
%\setcounter{MaxMatrixCols}{40} %Matrices con más columnas
%\usepackage{bm} %Comando \bm{}, negritas en textos matemáticos

\PassOptionsToPackage{hyphens}{url} %Para que los enlaces no se salgan del borde. Poner \sloppy\url{}
\usepackage[colorlinks=true,linkcolor=black,urlcolor=blue,citecolor=blue]{hyperref} %Hipervínculos, así como enlazar partes, citas, etc del mismo documento. Para enlaces: \href{https://www.youtube.com/}{Youtube}

\usepackage{multicol} %Para dividir el texto en columnas
\usepackage{titlesec}
\titleformat{\section}[block]{\large\bfseries\centering}{\thesection.}{1mm}{} %Centra las secciones y las deja bonitas. REQUIERE TITLESEC
\titleformat{\subsection}[block]{\bfseries\centering}{\thesubsection.}{1mm}{}
\usepackage[left=2cm,right=2cm,top=2.5cm,bottom=3cm]{geometry} %Limitar bordes. Puede ponerse también con lmargin, rmargin.
\parindent =0cm %Cambiar todas las sangrías
\usepackage{fancyhdr} %Un formato guay para papers
\pagestyle{fancy} %Que use dicho formato
\fancyhead{} %Eliminar definiciones previas 
\fancyhead[c]{TITULO}
\fancyhead[r]{Universidad de Nosedonde}
\fancyfoot{}
\fancyfoot[r]{\thepage} %El comando \thepage pone la página
\fancyfoot[l]{AUTOR}
\renewcommand{\headrulewidth}{0.9pt} %Cambiar grosor línea arriba, o eliminarla (0pt)
\renewcommand{\footrulewidth}{0.5pt} %Crear y editar línea de abajo

\begin{document}
\begin{titlepage} %Primera página aparte con el título
\begin{center} %Para centrar el texto
\vspace*{2\baselineskip} %Avanza filas. Si está al principio, hay que poner *
\hrule height 3pt
\vspace*{0.5\baselineskip}
{\Huge \textbf{Técnicas experimentales en *algo* *numeritos romanos*}}\\[0.1cm]
{\large \textbf{UNIVERSIDAD DE NOSEDONDE}}
\vspace*{0.5\baselineskip}
\hrule
\vspace*{6\baselineskip}
\includegraphics[width=7cm]{fotos/escudo}\\
\vspace*{4.5\baselineskip}
{\Huge \textbf{TÍTULO\\Y TÍTULO:\\[0.5cm]MÁS TÍTULO\\}}
\vfill %Desplaza hasta el final de la página
\textit{AUTOR}\\
%\today %Muestra la fecha
\end{center}
\end{titlepage}

\tableofcontents
\thispagestyle{empty} %Quitar el formato en esta página
\newpage
\setcounter{page}{1} %Que no empiece la primera página en el índice

\renewcommand{\abstractname}{Abstract} %Para que lo llame Abstract, no Resumen
\addcontentsline{toc}{section}{Abstract}%Para mostrar el abstract en el índice. Esta línea ha de ponerse junto al contenido a listar.
\begin{abstract}
En el presente estudio se tratará de hacer algo y este es el abstract
\end{abstract}
\begin{multicols}{2} %Ponemos 2 columnas. Hay que cerrarlo al final. Si queremos que no empiece una hasta acabar la anterior, ponemos \begin{multicols*}

\section{Fundamento teórico}
\subsection{Hipótesis de Planck}
Planck\footnote{footnote de Planck} hizo el dibujo \eqref{etiqueta}.

\begin{figure}[H]
\centering
\includegraphics[width=3cm]{fotos/escudo}
\caption{explicación dibujo \cite{referencia}}
\label{etiqueta}
\end{figure}

Para poner fórmulas en medio del texto se hace ${\vartriangle}\varepsilon$ y para ponerlas separadas se hace como en la ecuación \eqref{ecuacion}
\begin{equation}
\Delta\varepsilon=h\nu
\label{ecuacion}
\end{equation} % {equation*} si no queremos numerarla


\section{Realización experimental}
\subsection{Montaje del experimento}
dibujitos noseque

\subsection{Consideraciones previas}
\label{consideraciones}
\begin{itemize}
\item elemento de lista 1
\item elemento de lista 2
\end{itemize}

\subsection{Realización del experimento}
blablabla
\section{Medidas y análisis de datos}
\subsection{Interpretación de resultados}
El experimento ha sido un éxito
\subsection{Fuentes de errores}
Pues hay muchos errores

\end{multicols} %Se acaban las dobles columnas
\section{Conclusiones}
Las conclusiones son blablabla

\newpage
%\renewcommand{\refname}{Bibliografía} %Para que ponga eso en vez de referencias
\addcontentsline{toc}{section}{Referencias} %Para mostrar las referencias en el índice. Esta línea ha de ponerse junto al contenido a listar.
\begin{thebibliography}{99} %99 para que no de problemas de limitación
\bibitem{referencia} Albert Einstein, Un punto de vista heurístico sobre la producción y transformación de la luz, \textit{Annalen der Physik} \textbf{17}, 132-148 (1905)
\end{thebibliography}
\end{document}
